\documentclass[12pt]{article}
\usepackage[dvipsnames]{xcolor}
\usepackage{hyperref, pagecolor, mdframed }
\usepackage{tabularx, graphicx, amsmath, latexsym, amsfonts, amssymb, amsthm,
amscd, geometry, xspace, enumerate, mathtools}
\usepackage{tikz}

\theoremstyle{plain}
\newtheorem{thm}[subsubsection]{Th\'eor\`eme}
\newtheorem{lem}[subsubsection]{Lemme}
\newtheorem{cor}[subsubsection]{Corollaire}

\theoremstyle{definition}
\newtheorem{defn}[subsubsection]{Definition}
\newtheorem{prop}[subsubsection]{Proposition}

\newcommand{\fdiv}{\textrm{div}}
\newcommand{\Z}{\mathbb{Z}}
\newcommand{\N}{\mathbb{N}}
\newcommand{\Q}{\mathbb{Q}}
\newcommand{\F}{\mathbb{F}}
\newcommand{\algK}{\overline{K}}
\newcommand{\algF}{\overline{\mathbb{F}}}
\newcommand{\Pic}{\textrm{Pic}}
\newcommand{\Hom}{\textrm{Hom}}
\newcommand{\End}{\textrm{End}}
\newcommand{\Disc}{\textrm{Disc}}
\newcommand{\Det}{\textrm{Det}}
\newcommand{\Tr}{\textrm{Tr}}
\newcommand{\Or}{\mathcal{O}}
\newcommand{\OK}{\mathcal{O}_{K}}
\newcommand{\OL}{\mathcal{O}_{L}}
\newcommand{\C}{\mathbb{C}}
\newcommand{\ai}{\mathfrak{a}}
\newcommand{\bi}{\mathfrak{b}}
\newcommand{\w}{\omega}
\newcommand{\gr}{\color{Sepia}}
\newcommand{\rg}{\color{Red}}
\hypersetup{
    colorlinks=true,
    linkcolor=blue,
    urlcolor=Green,
    filecolor=RoyalPurple
}

\newcolumntype{M}[1]{>{\raggedright}m{#1}}
\definecolor{wgrey}{RGB}{148, 38, 55}


\begin{document}
\title{Produit d'idéaux}
\maketitle
Si $I=(a, b+id)$ et $J=(c, f+ig)$ on les représente via une forme normale de hermite:
\[
    I=\begin{pmatrix}a & b\\ 0&d\end{pmatrix}
\]
\[
    J=\begin{pmatrix}c & f\\ 0&g\end{pmatrix}
\]

Pour calculer leur produit, on regarde l'idéal produit 
\[
    IJ=(ac, af+iag, cb+icd, bf-dg+i(cd+bg))
\]
Qu'on représente en:
\[
    IJ=\begin{pmatrix}ac & af & cb & bf-dg\\
        0 & ag & cd & cd+bg\\
    \end{pmatrix}
\]
Dont on calcule la forme normale de hermite pour obtenir 
\[IJ=(N(IJ), x+if)\]

(Tout est transposé dans le code car FLINT effectue la réduction 
hnf sur les lignes)
\end{document}