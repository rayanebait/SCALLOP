\documentclass[12pt]{article}
\usepackage[dvipsnames]{xcolor}
\usepackage{hyperref, pagecolor, mdframed }
\usepackage{tabularx, graphicx, amsmath, latexsym, amsfonts, amssymb, amsthm,
amscd, geometry, xspace, enumerate, mathtools}
\usepackage{tikz}

\theoremstyle{plain}
\newtheorem{thm}[subsubsection]{Th\'eor\`eme}
\newtheorem{lem}[subsubsection]{Lemme}
\newtheorem{cor}[subsubsection]{Corollaire}

\theoremstyle{definition}
\newtheorem{defn}[subsubsection]{Definition}
\newtheorem{prop}[subsubsection]{Proposition}

\newcommand{\fdiv}{\textrm{div}}
\newcommand{\Z}{\mathbb{Z}}
\newcommand{\N}{\mathbb{N}}
\newcommand{\Q}{\mathbb{Q}}
\newcommand{\F}{\mathbb{F}}
\newcommand{\algK}{\overline{K}}
\newcommand{\algF}{\overline{\mathbb{F}}}
\newcommand{\Pic}{\textrm{Pic}}
\newcommand{\Hom}{\textrm{Hom}}
\newcommand{\End}{\textrm{End}}
\newcommand{\Disc}{\textrm{Disc}}
\newcommand{\Det}{\textrm{Det}}
\newcommand{\Tr}{\textrm{Tr}}
\newcommand{\Or}{\mathcal{O}}
\newcommand{\OK}{\mathcal{O}_{K}}
\newcommand{\OL}{\mathcal{O}_{L}}
\newcommand{\C}{\mathbb{C}}
\newcommand{\ai}{\mathfrak{a}}
\newcommand{\bi}{\mathfrak{b}}
\newcommand{\w}{\omega}
\newcommand{\gr}{\color{Sepia}}
\newcommand{\rg}{\color{Red}}
\hypersetup{
    colorlinks=true,
    linkcolor=blue,
    urlcolor=Green,
    filecolor=RoyalPurple
}

\newcolumntype{M}[1]{>{\raggedright}m{#1}}
\definecolor{wgrey}{RGB}{148, 38, 55}


\begin{document}
\title{Explication du code}
\maketitle
\section{produit d'idéaux}
(Cette partie est finalement pas utilisée telle quelle, la fonctionnalité étant implémentée en sage)
Si $I=(a, b+id)$ et $J=(c, f+ig)$ on les représente via une forme normale de hermite:
\[
    I=\begin{pmatrix}a & b\\ 0&d\end{pmatrix}
\]
\[
    J=\begin{pmatrix}c & f\\ 0&g\end{pmatrix}
\]

Pour calculer leur produit, on regarde l'idéal produit 
\[
    IJ=(ac, af+iag, cb+icd, bf-dg+i(cd+bg))
\]
Qu'on représente en:
\[
    IJ=\begin{pmatrix}ac & af & cb & bf-dg\\
        0 & ag & cd & cd+bg\\
    \end{pmatrix}
\]
Dont on calcule la forme normale de hermite pour obtenir 
\[IJ=(N(IJ), x+if)\]

(Tout est transposé dans le code car FLINT effectue la réduction 
hnf sur les lignes)

\section{Calcul du réseau de relations}
Les idéaux d'ordres, le calcul de produit et la composition de forme quadratiques n'étant pas implémentés 
en sage. Je l'ai fait en C à l'aide de FLINT.
On utilise l'algorithme de Pohlig-Hellman couplé à rho-pollard. L'implémentation 
est faite à l'aide du module qfb de FLINT qui permet de calculer dans le groupe de classe de 
formes quadratiques binaires de discriminant donné.

\section{Paramètres}
Les candidats sont calculés à l'aide de 
gen\_conductor\_choices dans lib/ideals. Chaque 
fichier candidate\_conductorsN contient les 
candidats pour les N premiers nombres 
premiers décomposés dans $\Z[i]$. L'évaluation 
de chaque candidats est éffectué par 
eval\_candidate présent dans lib/eval\_candidate
à l'aide de ECM avec abandon et 30 processus 
en parallèle qui sont terminés au bout d'une seconde 
si ils n'ont pas finis. 

\subsection{Paramètre de 40 et 80 bits}
Pour $n_1+n_2=11$ j'ai pris 
\[\alpha=3014688773870022715669219i + 73018318326246924528693954\]
avec $f\approx 2^{81}$ et $f- (\frac{-1}{f})$ 
qui est $2^{30}$-lisse. J'ai aussi pris 
\[p=31392239785933786038660665604566479\]\[=cL-1\]
 avec $c=16$.
Pour $n_1+n_2=17$ j'ai pris 
\[\alpha=606346906079138499752787264655000342496041920427i\]
\[~~+ 375198466882833042822684243162077125238505408858\]
avec $f\approx 2^{161}$ et $f-(\frac{-1}{f})$
qui est $2^{36}$-lisse. J'ai aussi pris \[
    p=873224592283478872608679328148373760813822474964590896670059
\]
\[=cL-1\] avec $c=28$.

La table des logs discrets est txt/dlogs\_N\_bits.md.

\section{Commandes}
Pour lancer le calcul du réseau de relation, depuis 
le dossier Clib:
\begin{itemize}
    \item make
    \item ./bin/lattice\_relations\\ ../txt/conductor\_N\_bits.md ../txt/sqrts\_M\_primes.md ../txt/dlogs\_N\_bits.md
\end{itemize}

Pour $N=40$ mettre $M=11$ et pour $N=80$ mettre $M=17$.

\end{document}