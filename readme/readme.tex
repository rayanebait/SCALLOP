\documentclass[12pt]{article}
\usepackage[dvipsnames]{xcolor}
\usepackage{hyperref, pagecolor, mdframed }
\usepackage{tabularx, graphicx, amsmath, latexsym, amsfonts, amssymb, amsthm,
amscd, geometry, xspace, enumerate, mathtools}
\usepackage{tikz}

\theoremstyle{plain}
\newtheorem{thm}[subsubsection]{Th\'eor\`eme}
\newtheorem{lem}[subsubsection]{Lemme}
\newtheorem{cor}[subsubsection]{Corollaire}

\theoremstyle{definition}
\newtheorem{defn}[subsubsection]{Definition}
\newtheorem{prop}[subsubsection]{Proposition}

\newcommand{\fdiv}{\textrm{div}}
\newcommand{\Z}{\mathbb{Z}}
\newcommand{\N}{\mathbb{N}}
\newcommand{\Q}{\mathbb{Q}}
\newcommand{\F}{\mathbb{F}}
\newcommand{\algK}{\overline{K}}
\newcommand{\algF}{\overline{\mathbb{F}}}
\newcommand{\Pic}{\textrm{Pic}}
\newcommand{\Hom}{\textrm{Hom}}
\newcommand{\End}{\textrm{End}}
\newcommand{\Disc}{\textrm{Disc}}
\newcommand{\Det}{\textrm{Det}}
\newcommand{\Tr}{\textrm{Tr}}
\newcommand{\Or}{\mathcal{O}}
\newcommand{\OK}{\mathcal{O}_{K}}
\newcommand{\OL}{\mathcal{O}_{L}}
\newcommand{\C}{\mathbb{C}}
\newcommand{\ai}{\mathfrak{a}}
\newcommand{\bi}{\mathfrak{b}}
\newcommand{\w}{\omega}
\newcommand{\gr}{\color{Sepia}}
\newcommand{\rg}{\color{Red}}
\hypersetup{
    colorlinks=true,
    linkcolor=blue,
    urlcolor=Green,
    filecolor=RoyalPurple
}

\newcolumntype{M}[1]{>{\raggedright}m{#1}}
\definecolor{wgrey}{RGB}{148, 38, 55}


\begin{document}
\title{Explication du code}
\maketitle

\section{Calcul du réseau de relations}
Je l'ai fait en C à l'aide de FLINT. Je n'étais pas conscient du plongement dans le tore.
J'utilise l'algorithme de Pohlig-Hellman couplé à un rho-pollard basique. L'implémentation 
est faite à l'aide du module qfb de FLINT qui permet de calculer dans le groupe de classe de 
formes quadratiques binaires de discriminant donné.

\section{Paramètres}
Les candidats sont calculés à l'aide de 
gen\_conductor\_choices dans lib/candidate\_conductors. Chaque 
fichier candidate\_conductorsN contient les 
candidats pour les N premiers nombres 
premiers décomposés dans $\Z[i]$. L'évaluation 
de chaque candidats est éffectué par 
eval\_candidate présent dans lib/eval\_candidate
à l'aide de ECM avec abandon et plusieurs processus 
en parallèle qui sont terminés au bout d'un temps donné 
si ils n'ont pas finis. 

\subsection{Des paramètres}
Les conducteurs, leurs factorisations et d'autres données sont compilées pour la lecture dans:
\begin{center}
  lib/eval\_candidates/conductor\_dataN.
\end{center}
Le dossier txt regroupe les données utilisées par le code.
\subsection{Paramètres trouvés}
Pour $n_1+n_2=3$ il y'a $\alpha=109i-482$ ou $f-1=2*2*3*3*3$ (pour un exemple jouet).\\
Pour $n_1+n_2=14$ j'ai trouvé
\[\alpha=18359253140637317346421*i+51954880756346626090702\]
avec $f\approx 2^{74}$ et $f- (\frac{-1}{f})$ 
qui est $2^{15}$-lisse et $L(f,1/2)\approx 2**20$! J'ai pris
\[p=
4435728726000669680627459
\]\[=cL-1\]
 avec $c=36$.
Pour $n_1+n_2=19$ j'ai trouvé
\[\alpha=3787463183160155300151628190651699i
\]
\[~~+4029106655575753933813779245328898
\]
avec $f\approx 2^{111}$ et $f-(\frac{-1}{f})$
qui est $2^{24}$-lisse. J'ai aussi pris \[
    p= 148379836973137677914241375224586059
    \]
\[=cL-1\] avec $c=12$.

La table des logs discrets est txt/dlogs\_N\_primes.md.

\section{Commandes}
Pour lancer le calcul du réseau de relation, depuis 
le dossier Clib:
\begin{itemize}
    \item make
    \item ./bin/lattice\_relations\\ ../txt/conductor\_N\_primes.md ../txt/sqrts\_N\_primes.md ../txt/dlogs\_N\_primes.md
\end{itemize}

Actuellement, seulement $N=3, 14,19, 20$ sont utilisables (il y'a d'autres bons candidats pour $N=26,27$)\\


Depuis le dossier lib:
\begin{itemize}
    \item candidate\_conductors et
    eval\_candidate contiennent les fonctions
     permettant de générer et évaluer les conducteurs
\end{itemize}
Pour les autres, chaque fonction appelée avec $-h$ pour 
afficher les options. La plupart on $-v$ pour afficher 
les calculs faits et $-n$ pour faire les calculs sur 
le conducteur choisi pour $n$ nombres premiers. Parmis 
ces fonctions:
\begin{itemize}
    \item gen\_curve et endo permettent respectivement
    de générer la courbe et de calculer un endomorphisme
    de norme $M>p$.
    \item gen\_prime génère un premier associé à un conducteur 
\end{itemize}

\end{document}